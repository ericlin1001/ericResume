% !TEX program = xelatex
% !TEX encoding = UTF-8 Unicode
% !Mode:: "TeX:UTF-8"


\documentclass{resume}
\newcommand\quelle[1]{{%
		\unskip\nobreak\hfil\penalty50
		\hskip2em\hbox{}\nobreak\hfil #1%
\parfillskip=0pt \finalhyphendemerits=0 \par}}

\usepackage{zh_CN-Adobefonts_external} % Simplified Chinese Support using external fonts (./fonts/zh_CN-Adobe/)
%\usepackage{zh_CN-Adobefonts_internal} % Simplified Chinese Support using system fonts
\usepackage{linespacing_fix} % disable extra space before next section

\begin{document}
\pagenumbering{gobble} % suppress displaying page number

\name{林俊浩}

\basicInfo{
	\email{ericlin1001@qq.com} \textperiodcentered\ 
	\phone{(+86) 135-8053-2764} 
	%\textperiodcentered\ 
	%\linkedin[billryan8]{https://www.linkedin.com/in/billryan8}}
}
\basicInfo{
	\faPaw\ GitHub: \url{https://github.com/ericlin1001}
}

\section{\faGraduationCap\ 教育背景}
\datedsubsection{\textbf{中山大学}, 广州}{2015年9月 -- 至今}
\textit{在读硕士研究生}\  计算机科学与技术, 预计 2018年6月 毕业
\datedsubsection{\textbf{中山大学}, 广州}{2011年9月 -- 2015年6月}
%TODO::本科 or 学士
\textit{学士}\ 计算机科学与技术 

%\section{\faMale\ 校内经历}
%\datedsubsection{\textbf{中山大学}, 广州}{2015 -- 至今}

\section{\faUsers\ 项目经历}
\datedsubsection{\textbf{自适应并行差分算法}}{2015年3月 -- 2016年7月}
% TODO:: Is Maintainer the exact role.
\role{C++, MPI}{个人学术项目}
%A parallel program, for academic purpose, \url{https://github.com/ericlin1001/AdaptivePDE}
一个运行在集群上的并行算法程序, 用来解决功率电子优化问题(PEC), \url{https://github.com/ericlin1001/AdaptivePDE}
\begin{itemize}
		%\item Implemented parallel Differential Evolution(DE) algorithm
	\item 实现差分进化算法的并行化, 并行加速比(speedup)约等于CPU核心个数的1/2
		%\item Defined a Json format for setting up DE
	\item 支持在多个CPU核心之间自适应分配任务, 高效利用了CPU核心的资源
	\item 提出了一种Json格式的参数表示, 使得算法参数设置有一个统一的标准
		%\item Support for adaptively distributing tasks among cores
	\item 在GECCO '16 Companion发表了相应论文: \textit{Parallel Differential Evolution Based on Distributed Cloud Computing Resources for Power Electronic Circuit Optimization} %\quelle{2016年7月}
\end{itemize}

%\datedsubsection{\textbf{远程控制软件}}{2016年3月 -- 2016年4月}
%\role{Python}{个人项目}
%一个采用Client-Server模式的远程控制软件, \url{https://github.com/ericlin1001/RemoteControl}
%\begin{itemize}
%	\item 实现了实时的屏幕捕捉, 达到人眼无法分辨的延迟
%	\item 支持鼠标和键盘的同步
%\end{itemize}

%\datedsubsection{\textbf{Help-Each-Other}}{Jul. 2013 -- Dec. 2013}
\datedsubsection{\textbf{互帮互助系统开发}}{Jul. 2013 -- Dec. 2013}
%\role{Php, Manager}{Team Projects, collaborated with three classmates}
\role{Php, 负责人}{团队项目, 和另外三个同学合作开发}
%A web application providing a way to facilitate people help each other, \url{https://github.com/ericlin1001/HelpEachOtherProject}
一个web端应用, 支持用户发布需求和帮助它人, \url{https://github.com/ericlin1001/HelpEachOtherProject}
\begin{itemize}
		%	\item Implemented task management system, i.e. posing and accepting tasks.
	\item 实现了任务的发布和接收系统, 使用户可以随时发布需求和帮助他人	

		%	\item Support for chatting between different users.
	\item 实现了完整线上聊天系统
		%\item Ranking list was implemented.
	\item 支持用户实时排名, 意见反馈
		%\item Support for feedback.
\end{itemize}

\datedsubsection{\textbf{操作系统开发}}{2013年1月 -- 2013年6月}
\role{Asm, C}{课程大作业, 个人项目}
%A basic Operating System(OS), \url{https://github.com/ericlin1001/EricOSProject}
一个基本的操作系统, \url{https://github.com/ericlin1001/EricOSProject}
\begin{itemize}
		% write a readme for the project.
		%	\item Implemented Buddy based Memory Management, FAT12 Filesystem
	\item 实现了基于Buddy的内存管理, 支持内存动态分配 
	\item 支持FAT12文件系统的文件读入
		%	\item Implemented Process Scheduling, Inter-process Communication, and IO
	\item 实现了进程调度, 进程间通信, 标准输入输出
	\item 此代码被老师作为之后实验课的标准, 并被老师邀请上去代为讲解
\end{itemize}

% Reference Test
%\datedsubsection{\textbf{Paper Title\cite{zaharia2012resilient}}}{May. 2015}
%An xxx optimized for xxx\cite{verma2015large}
%\begin{itemize}
% \item main contribution
%\end{itemize}

\section{\faCogs\ IT 技能}
\begin{itemize}[parsep=0.5ex]
	\item 编程语言: C++/C, Python, Php, Asm, Java, Matlab
	\item 其它技能: Linux, Git, MySQL, Json, HTML, XML, Autoit, LAMP
\end{itemize}

\section{\faHeartO\ 获奖情况}

%\datedline{Published \textit{Parallel Differential Evolution Based on Distributed Cloud Computing Resources for Power Electronic Circuit Optimization} in GECCO '16 Companion}{Jul. 2016}
%\par Published \textit{Parallel Differential Evolution Based on Distributed Cloud Computing Resources for Power Electronic Circuit Optimization} in GECCO '16 Companion \hfill Jul. 2016 \par
%#OK: \textit{Publish Parallel Differential Evolution Based on Distributed Cloud Computing Resources for Power Electronic Circuit Optimization} in GECCO '16 Companion \hfill {Jul. 2016}
\par
\datedline{2015-2016 中山大学优秀研究生奖学金, 前 10\%}{2016年6月}
	\datedline{2011-2012 中山大学优秀学生二等奖学金, 前 15\%}{2012年6月}
		%\datedline{中山大学镇泰奖学金}{2012年6月}


		\section{\faInfo\ 其它}
		\begin{itemize}[parsep=0.5ex]
			\item 职务: 班长(2015年-至今), 助教(2016年-2017年)
				%\item 技术博客: \url{https://ericlin1001.github.io}
				%	\item GitHub: \url{https://github.com/ericlin1001}
			\item 语言: 英语 - 熟练(CET-6 553分)
				%\item 爱好: 阅读, 滑板, 乒乓球, 跑步
		\end{itemize}

		%% Reference
		%\newpage
		%\bibliographystyle{IEEETran}
		%\bibliography{mycite}
		\end{document}
